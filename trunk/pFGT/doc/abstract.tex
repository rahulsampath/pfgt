We present fast adaptive parallel algorithms to compute the sum of $N$ Gaussians at $N$ points. 
Direct sequential computation of this sum would take $\mathcal{O}(N^2)$ time. The parallel time complexity estimates for our algorithms are $ \mathcal{O} \left(\frac{N}{n_p} \right)$ for uniform point distributions and $ \mathcal{O} \left(\frac{N}{n_p}\log \frac{N}{n_p} + n_p \log n_p\right)$ for nonuniform distributions using $n_p$ CPUs. We incorporate a plane-wave representation of the Gaussian kernel which permits ``diagonal translation''. We use parallel octrees and a new scheme for translating the plane-waves to efficiently handle nonuniform distributions. Computing the transform to six-digit accuracy at 120 billion points took approximately 140 seconds using 4096 cores on the Jaguar supercomputer at the Oak Ridge National Laboratory. 

Our implementation is {\em kernel-independent} and can handle other ``Gaussian-type'' kernels even
 when an explicit analytic expression for the kernel is not known. These algorithms form a new class of core computational machinery for solving parabolic PDEs on massively parallel architectures. 