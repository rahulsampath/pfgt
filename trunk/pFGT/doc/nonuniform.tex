In the uniform distribution case, one can precisely estimate the threshold $n^*$ that determines whether one should use the truncation or the expansion algorithm. However, when the source and target distributions are highly nonuniform, as is the case in most practical applications, it is not straightforward. For example, when we superimpose a regular grid structure of FGT on a nonuniform distribution, some boxes will have lot of points while some are almost empty. 

In this section, we present a scalable algorithm for nonuniform distributions based on the ideas introduced in \cite{veerapaneni08}. A main ingredient of our alogirithm is building an \emph{octree} from the given set of points% 
%
\footnote{The original FGT \cite{fgt} handles nonuniform distributions without using any sort of hierarchical data-structures. In principle, one can extend these ideas to the present context. However there are certain advantages of using octrees both at the implementational and algorithmic levels. [NEED TO EXPAND]}. 
%
We discuss this next.
%

{\em W2D.} At this stage, all processors have finished forming the plane-wave expansions at their respective boxes. 
Now, we use this information to evaluate the influence of $T_e$ on $T_d$. Each box $B \in T_e$ must visit every target point $x \in T_d$ in its interaction list and evaluate the Gauss transform using
%
\beq F(x) = \sum_{|k| \leq p} e^{-\frac{\norm{z_k}^2}{4}} e^{i z_k \cdot (x - c^B)/\sqrt{\delta}} w_k \eeq
%
We perform a two-step communication to achieve this. In the first step, we send requests to the owners of boxes that are in the interaction lists of targets in $T_d$. In the second step, the owners send the plane-wave coefficients \footnote{ If $T_d$ receives any requests, it simply sends zero plane-wave coefficients. This can be avoided by an extra communication step}. 

\hfill $\bigO(p^3 N / n_p)$
% assuming O(N) points in the direct tree

{\em D2D.}


{\em D2L.} This step is a dual of the W2D. Here, we compute the influence of $T_d$ on $T_e$ by visiting every box $D \in T_e$ in the interaction list of a source $y \in T_d$ and modifying the local expansion as 
%
\beq v_k += \, f(y) e^{i z_k \cdot (c^D - y)/\sqrt{\delta}} \eeq
%


\subsection{Parallel octrees}
We assume sources and targets are the same for simplicity. We also assume that the octree is constructed so that there are no more than a fixed number of points in each leaf node. 

\subsection{FGT on octrees}


\begin{algorithm}[!h]
\caption{{\em Tree Splitting}}
{\tt
\begin{algorithmic}
\STATE
  \FOR {each leaf node $\ell$}
      \IF {$|\ell| > \sqrt{\delta}$}
          \STATE assign $\ell$ to $T_d$ (direct or truncation based)
      \ELSE
          \STATE assign $\ell$ to $T_e$ (expansion based)
      \ENDIF
  \ENDFOR
\STATE
\end{algorithmic}
}
\end{algorithm}


Then the following are the different stages of the parallel algorithm:

{\em (i) S2W}. In this step, we form the plane-wave expansions at the set of boxes that cover $T_e$. 
We visit each leaf, and add its contribution to the box it is contained in or to all the boxes it contains. 

\hfill $\bigO(N p^3/n_p)$

{\em (ii) S2W-Comm.} Because the boxes are partitioned by [RAHUL] and the octree is partitioned by Morton ordering, it is not necessary that the boxes contained in a leaf (or vice-versa) belong to the same processor. Therefore, after the S2W step, we send the plane-wave expansions formed in each processor to the owner of the respective boxes. Then, we visit each box $B \in T_e$ and add all the plane-wave expansions formed at different processors to obtain $\{w_k \}_{|k| \leq p}$.  

\hfill $\bigO(K^3 n_p)$

{\em (iii) W2D.} At this stage, all processors have finished forming the plane-wave expansions at their respective boxes. 
Now, we use this information to evaluate the influence of $T_e$ on $T_d$. Each box $B \in T_e$ must visit every target point $x \in T_d$ in its interaction list and evaluate the Gauss transform using
%
\beq F(x) = \sum_{|k| \leq p} e^{-\frac{\norm{z_k}^2}{4}} e^{i z_k \cdot (x - c^B)/\sqrt{\delta}} w_k \eeq
%
We perform a two-step communication to achieve this. In the first step, we send requests to the owners of boxes that are in the interaction lists of targets in $T_d$. In the second step, the owners send the plane-wave coefficients \footnote{ If $T_d$ receives any requests, it simply sends zero plane-wave coefficients. This can be avoided by an extra communication step}. 

\hfill $\bigO(p^3 N / n_p)$
% assuming O(N) points in the direct tree

{\em (iv) D2D.}
1. For each point (p) within each octant marked as Direct, compute the
point (A) in the -ve corner and the point (B) in the +ve corner of p's
interaction list. Note, the Morton ids of all points in the interaction
list of point p will be >= that of point A and <= that of point B.
2. Gather the Morton id of the first (in the Morton sorted list) direct
octant on each processor. This will give the smallest Morton id of
direct octants on each processor. Handle the case where some processors
do not contain any direct octants.
3. Figure out the processor with the greatest value <= A in the list
computed in step 2.
4. Figure out the processor with the greatest value <= B in the list
computed in step 2.
5. Send point p and the corresponding source f to all processors that
lie between the processors computed in step 3 and step 4. Note, we are 
using the property that if the Morton id of Octant C < Morton id of 
Octant D then the Morton id of any point within Octant C is >  Morton
id of Octant C and is < Morton id of Octant D.
6. For each point p recieved in step 5, compute the
point (A) in the -ve corner and the point (B) in the +ve corner of p's
interaction list. Find the direct octant O1 with the maximum Morton id that
is <= A and the direct octant O2 with the maximum Morton id that is <=
B. Loop over the set of Direct octants O with Morton ids >= that of O1
and <= that of O2. Loop over the set of points within O that lie in the
interaction list of p. For each of these points, add the contribution to
the Gauss transform from p.

\hfill $\bigO(p^3 N/n_p)$

{\em (v) W2L.} To allow execution of the sweeping algorithm independently on each processor, we exchange the plane-wave expansions of $\lfloor K/2 \rfloor ^3$ boundary boxes across adjacent processors. 

\hfill $\bigO(15 p^3 |B|/n_p + K^3 p^3 n_p)$
% doin the sweep + communicating ghost values

{\em (vi) D2L.} This step is a dual of the W2D. Here, we compute the influence of $T_d$ on $T_e$ by visiting every box $D \in T_e$ in the interaction list of a source $y \in T_d$ and modifying the local expansion as 
%
\beq v_k += \, f(y) e^{i z_k \cdot (c^D - y)/\sqrt{\delta}} \eeq
%
This requires a one-step communication: in each processor, we visit every source $y \in T_d$ 
and send its information ($y$ and $f(y)$) to processors that own boxes that are in $\mathcal{I}[y]$.

\hfill $\bigO(p^3 N / n_p)$

{\em (vii) L2T-Comm.} This step is a dual of S2W-Comm. Here, we send the local expansions formed at boxes in each processor 
to the processors that own leaf nodes that are contained within this FGT box. 

\hfill $\bigO(K^3 n_p)$

{\em (viii) L2T.} At this stage, all boxes must have formed their local expansions and $F(x)$ must have been computed at the points $x \in T_d$. The only remaining step is to compute $F(x)$ for points in $x \in T_e$ by using (\ref{eqn:l2t}) which can be done independently in each processor. 

\hfill $\bigO(N p^3/n_p)$

\begin{algorithm}[!h]
\caption{\em FGT on a split tree}
%\label{a:ofgt}
{\tt
\begin{algorithmic}
\STATE
%\STATE {\sc FGT on a split Octree}
  \FOR {each $\ell \in T_e$}
      \STATE (S2W) Add contribution of point in $\ell$ to the FGT box it belongs
  \ENDFOR
  \STATE
  \STATE (W2L) Form local expansions using sweeping
  \STATE 

  \STATE {\sc Evaluate the effect of all sources in $T_d$}
  \FOR {each $\ell \in T_d$}
       \FOR {$x \in \ell$}
          \STATE Add contribution of $x$ to all the target boxes (in $T_e$) and target points (in $T_d$)     
       \ENDFOR  
  \ENDFOR
  
  \STATE 
  \STATE {\sc Evaluate the effect of all sources in $T_e$}
     \FOR {each FGT box $B \in T_e$}
        \STATE Use local expansion for target points within $B$ 
        \STATE
        \STATE Use wave expansion for target points in $T_e$ that are within its interaction list
     \ENDFOR  
\STATE
\end{algorithmic}
}
\end{algorithm}

\subsection{Overlapping communication with computation} 

{\em [In the parallel case, the interaction between $T_d$ and $T_e$ can be done while processors are communicating other info.] }

