For simplicity, we assume that the points are uniformly distributed and that they reside within a unit cube.
The design of fast algorithms for (\ref{gt}) is strongly dependent on three independent parameter viz., number of sources $N$, the bandwidth $\delta$ and desired accuracy $\epsilon$. A Gaussian centered at a source location interacts with targets that are within its support. If there are fewer targets than a threshold value $n^*$, we use a simple {\em truncation algorithm}, otherwise, we use a {\em expansion algorithm}. The threshold  value depends on all three independent parameters and we will discuss its choice after introducing both algorithms. 

\subsection{Truncation algorithm} 
Since the kernel in (\ref{gt}) decays exponentially, we can simply truncate the sum to
%
\beq F(x_j) = \sum_{y_k \in \mathcal{I}[x_j]} \kernel f(y_k) \eeq
%
where $\mathcal{I}[x_j]$ is the interaction list which includes all the sources that are within a distance $\sqrt{\delta \ln (1/\epsilon)}$. Beyond this distance, a Gaussian centered at $x_j$ decays below $\epsilon$. The complexity of this algorithm is $\bigO(N \sqrt{\delta \ln(1/\epsilon)})$. This is a common technique used in the graphics community. When $\delta$ is large and/or high accuracy is required, the cost of this algorithm grows quadratically.  

\subsection{Expansion algorithm}
There are several variations of FGT based on hermite expansions \cite{fgt}, plane-wave expansions \cite{greengard98}, and Chebyshev expansions \cite{}. Broadly speaking, all these algorithms partition the domain into there are three main steps in all these variations: (i) compress the influence of sources into a few multipole-type moments, (ii) translate the moments across the domain and (iii) evaluate the moments at the targets. Typically, the translation costs tend to dominate in 
The plane-wave expansion version has certain advantages for problems in three or lower dimensions as the traslational cost is minimized to a large extent . The latter version is further improved in \cite{fggt} for volumetric data. Our implementation is based on \cite{fggt} and we summarize it here. 

Central to the expansion algorithm is the finite-term plane-wave representation of the kernel,
\beq G_\delta(\norm{x_j - y_k}) \approx \sum_{|k| \leq p} \hat{G}(k) e^{i \lambda k \cdot (x_j - y_k)}, \quad \lambda = \frac{L}{p\sqrt{\delta}}\eeq
where $k = (k_1, k_2, k_3)$ and the parameters $p$ and $L$ are determined by the required precision. $\hat{G}$ is the discrete Fourier transform of the kernel. For the Gaussian kernel, we have 
\beq \hat{G}(k) = \left(\frac{L }{2p\sqrt{\pi}}\right)^3 e^{-\frac{\lambda^2 |k|^2 \delta}{4}}.\eeq

The algorithm begins by partitioning the domain into uniform boxes of size $\sqrt{\delta}$ each. A Gaussian located at the center of a box $B$ decays below $\epsilon$ beyond a fixed number of boxes. We call these boxes the interaction list of $B$, denoted by $\mathcal{I}[B]$. 

In the fast algorithm, a target point $x$ receives information from a source point $y$ via three stages:
\begin{description}
\item[\textbf{S2W}] The influence of all the sources in a box $B$ is condensed into a plane-wave expansion.
            \beq w_k = \sum_{y \in B} f(y) e^{i\lambda k \cdot (c^B - y)} \quad \forall |k| \leq p \eeq
\item[\textbf{W2L}] The plane-wave expansion of each box is transmitted to all the boxes in its interaction list. 
            \beq v_k = w_k e^{i\lambda k \cdot (c^D - c^B)}\eeq
\item[\textbf{L2T}] The local plane-wave expansion is evaluated at the target locations.
            \beq F(x) = \sum_{|k| \leq p} \hat{G}(k) v_k e^{i\lambda k \cdot (x - c^D)}\eeq
\end{description} 

The first step takes $\bigO(p^d )$

{\em Parallel implementation.} For simplicity, let us assume that $n_p = n^3$ for some positive integer $n$. Then, each processor gets has a regular grid of FGT boxes. Then, the S2W and L2T steps are embarassingly parallel. In S2W stage, each processor loops through its constituent boxes and forms the plane-wave expansions. Similarly each processor executes the L2T step independently. The cost for each step, therefore, is simply $\bigO(p^d N/n_p)$. The W2L step, on the other hand, requires translating the plane-wave expansions of $\bigO(K^d)$ boundary boxes across processors and hence the cost of this step is $\bigO(p^d K^d |B|/n_p + K^d n_p)$ where the second term is the communication cost. 

\subsection{Overall algorithm} 
	
The cost 
When combined with a sweeping algorithm for the W2L step, the translation step computational cost of the expansion algorithm is dominated by the  
Based on the cost of the expansion based algorithm, we 

{\tt
\begin{algorithmic}
\STATE
  \IF {$N \sqrt{\delta \ln \left( \frac{1}{\epsilon} \right)} < (2p)^3 $}
     \STATE Use truncation algorithm 
  \ELSE 
     \STATE Use expansion based algorithm
  \ENDIF
\STATE
\end{algorithmic}
}

----------------------
Once the wave expansions are formed at all the FGT boxes, the next step is to form each of their local expansions. Since the FGT boxes are of size $\bigO(\sqrt{\delta})$, only a fixed number of surrounding boxes contribute to the local expansion of a box $B$. We shall call these set of boxes as its {\em interaction list}, denoted by $\mathcal{I} [B]$. A {\em direct scheme} forms the local expansion by simply visiting all the boxes in $\mathcal{I}[B]$ and translating their wave expansions. After initializing the local expansions $\{ v_k \quad \forall \quad |k| \leq p \}$ to zero, the pseudo-code for the direct scheme is:
%
{\tt
\begin{algorithmic}
\STATE {\sc Direct scheme}
  \FOR {each $C \in \mathcal{I}[B]$}
           \STATE $ v_k^B \quad += e^{i z_k \cdot(c^B - c^C)/\sqrt{\delta}} w_k^C \quad \forall \quad |k| \leq p$
       \ENDFOR
\STATE
\end{algorithmic}
}
%
Assuming the size of $\mathcal{I}[B]$ is $K^3 - 1$, this algorithm requires $\bigO(K^3 p^3 N_B)$ work to form local expansions at all the boxes. 
