For simplicity, we assume that the points are uniformly distributed and that they reside within a unit cube.
The design of fast algorithms for (\ref{gt}) is strongly dependent on three independent parameters viz., the number of sources $N$, the bandwidth $\delta$ and desired accuracy $\epsilon$. A Gaussian centered at a source location interacts with targets that are within its support. If there are fewer targets than a threshold value $n^*$, we use a simple {\em truncation algorithm}, otherwise, we use a {\em expansion algorithm}. The threshold  value depends on all three independent parameters and we will discuss its choice after introducing both algorithms. 

\subsection{Truncation algorithm} 
Since the kernel in (\ref{gt}) decays exponentially, we can simply truncate the sum to
%
\beq F(x_j) = \sum_{y_k \in \mathcal{I}[x_j]} \kernel f(y_k) \eeq
%
where $\mathcal{I}[x_j]$ is the interaction list which includes all the sources that are within a distance $\sqrt{\delta \ln (1/\epsilon)}$. Beyond this distance, a Gaussian centered at $x_j$ decays below $\epsilon$. The complexity of this algorithm is $\bigO(N \sqrt{\delta \ln(1/\epsilon)})$. This is a common technique used in the graphics community (Gaussian blurring). When $\delta$ is large and/or high accuracy is required, the cost of this algorithm grows quadratically.  

\subsection{Expansion algorithm}
Broadly speaking, FGT algorithms \cite{fgt, greengard98} partition the domain into boxes of uniform size and proceed in three main steps: (i) compress the influence of sources into a few multipole-type moments, (ii) translate the moments across the domain and (iii) evaluate the moments at the targets. In the original FGT \cite{fgt}, the translation costs tend to dominate \cite{fggt}. The plane-wave expansion version \cite{greengard98} minimized the translation costs to a large extent but at the expense of higher computational cost for steps (i) and (iii). This problem is alleviated in \cite{fggt} to a large extent. Our implementation is based on \cite{fggt} and we summarize it here. 

Central to the expansion algorithm is the finite-term plane-wave expansion of the kernel,
\beq G_\delta(\norm{x_j - y_k}) \approx \sum_{|k| \leq p} \hat{G}(k) e^{i \lambda k \cdot (x_j - y_k)}, \quad \lambda = \frac{L}{p\sqrt{\delta}}\eeq
where $k = (k_1, k_2, k_3)$ and the parameters $p$ and $L$ are determined by the required precision. $\hat{G}$ is the discrete Fourier transform of the kernel. For the Gaussian kernel, it is given by
\beq \hat{G}(k) = \left(\frac{L }{2p\sqrt{\pi}}\right)^3 e^{-\frac{\lambda^2 |k|^2 \delta}{4}}. \label{eqn:ghat}\eeq

The algorithm begins by partitioning the domain into uniform boxes of size $\sqrt{\delta}$ each. A Gaussian located at the center of a box $B$ decays below $\epsilon$ beyond a fixed number of boxes. We call these boxes the interaction list of $B$, denoted by $\mathcal{I}[B]$. 

In the fast algorithm, a target point $x \in D$ receives information from a source point $y \in B$ via three stages:
\begin{description}
\item[\textbf{S2W}] The influence of all the sources in the box $B$ is condensed into a plane-wave expansion.
            \beq w_k = \sum_{y \in B} f(y) e^{i\lambda k \cdot (c^B - y)} \quad \forall\quad |k| \leq p \eeq
\item[\textbf{W2L}] The plane-wave expansion of each box $B$ is transmitted to every box $D$ in $\mathcal{I}[B]$. By superposition, the ``local'' plane-wave expansion of $D$ gets modified as
            \beq v_k += w_k e^{i\lambda k \cdot (c^D - c^B)} \label{e:w2l}\eeq
\item[\textbf{L2T}] The transform (\ref{gt}) is computed at each target by evaluating the local expansion of the box it is                          contained in.
            \beq F(x) = \sum_{|k| \leq p} \hat{G}(k) v_k e^{i\lambda k \cdot (x - c^D)} \label{eqn:l2t}\eeq
\end{description} 

First, the S2W step is executed seperately in each box with $\bigO(p^3 N)$ work. In a {\em direct scheme}, the W2L is executed by simply visiting each box $B$ and translating its plane-wave expasion to all the boxes in $\mathcal{I}[B]$. Assuming the size of interaction list is $K^3$, this algorithm requires $\bigO(K^3 p^3 |B|)$ work to form local expansions at all the boxes. Finally, the L2T is executed by visiting each box and evaluating the local expansion at the targets, requiring $\bigO(p^N)$ work. Therefore, the overall cost of a sequential algorithm is $\bigO(p^3 N + K^3 p^3 |B|)$.  

{\em Parallel implementation.} We create a regular grid of $|B|$ FGT boxes distributed on $n_p$ processors 
such that {\tt{(a)}} each box is owned by an unique processor and {\tt{(b)}} each processor owns a 
sub-grid of FGT boxes. For ease of implementation, we assume that $|B| > n_p$. 
 We use PETSc's \cite{petsc-user-ref, petsc-home-page} DA module to manage this distributed regular grid.
 The S2W and L2T steps are embarassingly parallel; in the former step, each processor independently forms the 
 plane-wave expansions for each box that it owns and in the latter step, each processor
 independently computes the transform (\ref{gt}) for each target contained within some box owned by
  that processor using the local expansion for that box. Hence, the cost for these two steps is 
  simply $\bigO(p^3 \frac{N}{n_p})$. On the other hand, the 
  W2L step will involve translating the plane-wave expansions of some boxes across processor boundaries 
  if the interaction lists of those boxes spread across multiple processors. Therefore, the computation 
  and communication costs for this step are $\bigO(p^3 K^3 \frac{|B|}{n_p})$ and
  $\bigO(K (\frac{|B|}{n_p})^{\frac{2}{3}} + K^2(\frac{|B|}{n_p})^{\frac{1}{3}} + K^3 )$, respectively. 

\subsection{Overall algorithm} 
The translation costs can be reduced dramatically by using the {\em sweeping algorithm} introduced in \cite{greengard98}. We present a modified version of the same in the next section. The total cost of the expansion algorithm, then, is typically dominated by the S2W and L2T steps which scale as $\bigO(p^3 N)$. However, when $\delta$ is lower and/or $\epsilon$ is higher, the cost of the expansion algorithm increases because condensing the sources into plane-waves becomes increasingly futile as the number sources per box gets reduced. On the other hand, the cost of the truncation algorithm decreases. Hence, the overall algorithm switches between the two algorithms and based on heuristics, we propose the following:

{\tt
\begin{algorithmic}
\STATE
  \IF {$N \sqrt{\delta \ln \left( \frac{1}{\epsilon} \right)} < (2p)^3 $}
     \STATE Use truncation algorithm 
  \ELSE 
     \STATE Use expansion algorithm
  \ENDIF
\STATE
\end{algorithmic}
}


%\subsection{Kernel independence} 
%The advantage of the plane-wave expansion (\ref{eqn:ghat}) is that once  
%If the kernel values are known at the discrete points $\{Lk/p \}_{|k| \leq p}$, then we can compute $\hat{G}$ using discrete Fourier transform. 
