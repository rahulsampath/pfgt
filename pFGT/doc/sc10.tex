%{{{ setup
\documentclass[conference]{IEEEtran}
\usepackage{amsmath,url,color,soul, amssymb}
\usepackage{setspace, graphicx}
\usepackage[x11names,rgb]{xcolor}
\usepackage{algorithm, algorithmic}
\usepackage{tikz}
\usetikzlibrary{snakes}
\usetikzlibrary{arrows}
\usetikzlibrary{shapes}
\usetikzlibrary{backgrounds}
\usetikzlibrary{patterns}
\usepackage{pgfplots}
\usepackage[cm]{fullpage}
\usepackage[scriptsize,tight]{subfigure}
\usepackage{url}
\usepackage[
pdftitle={title},
pdfcreator={pdftex},
pdfsubject={SCXY 2009},
hyperindex = {true},
colorlinks = {true},
linkcolor = {blue},
citecolor = {blue}
]{hyperref} 
\graphicspath{{figs/}}

\input macros.tex
\newcommand{\peq}{+\!\!=}
\newtheorem{conj}{Conjecture}[section]
\newtheorem{prob}{Problem}[section]

\usepgfplotslibrary{clickable}


\begin{document}

\title{Parallel Fast Gauss Transform}  


\author{\IEEEauthorblockN{Rahul S. Sampath}
\IEEEauthorblockA{Oak Ridge National Laboratory\\
       Oak Ridge, TN 37831\\
       Email: sampathrs@ornl.gov} 

\and
\IEEEauthorblockN{Hari Sundar}
\IEEEauthorblockA{Siemens Corporate Research \\
       Princeton, NJ 08540 \\
       Email: hari.sundar@siemens.com}

\and
\IEEEauthorblockN{Shravan K. Veerapaneni}
\IEEEauthorblockA{New York University \\
       New York, NY 10012 \\
       Email: shravan@cims.nyu.edu}
}
\date{}
\maketitle

\begin{abstract}
\input abstract.tex
\end{abstract}

\section{Introduction}  
\label{s:intro}
\input intro.tex
%
\section{Overview of FGT} 
\label{sc:fgt}
\input fgt.tex
%
\section{Translation scheme} 
\label{sc:sweep}
\input sweep.tex
%
\section{FGT on non-uniform distributions} 
\label{sc:nonuniform}
\input nonuniform.tex

\input parallelNonUniform.tex

\section{Results}
\label{sc:results}
\input results.tex

\section{Conclusions}
\label{sc:conclusions}
We will release the code under GPL in the near future. 

%\texbf{Periodic kernels.} If the kernel $G_\delta(x,y)$ is periodic in the unit cube, the fast Fourier transform (FFT) can be used for the fast computation of (\ref{gt}) assuming regular grids. 
%Several acceleration techniques for forming and evaluating plane wave expansion were introduced in \cite{fggt}. We will incorporate these in our final submission. When the sources come from a tensor product grid in each octant, the expansion constants can be reduced from exponential to linear in dimension. In the sequential case, this speed-up will make the cost of the algorithm comparable to that of fast Fourier transform (FFT). Since the parallel performance of FFT has been sub-optimal till date, we believe that our parallel FGT would be the method of choice even for regular grids. 



\section*{Acknowledgments}
\input acknowledge.tex

\bibliography{sc10}
\bibliographystyle{siam}
\end{document} 
