%{{{ setup
\documentclass[conference]{IEEEtran}
\usepackage{amsmath,url,color,algorithm, algorithmic, amssymb}
\usepackage{setspace, graphicx}
\usepackage[x11names,rgb]{xcolor}
\usepackage{tikz}
\usetikzlibrary{snakes}
\usetikzlibrary{arrows}
\usetikzlibrary{shapes}
\usetikzlibrary{backgrounds}
\usetikzlibrary{patterns}
\usepackage[cm]{fullpage}
\usepackage[scriptsize,tight]{subfigure}
\usepackage[
pdftitle={title},
pdfcreator={pdftex},
pdfsubject={SCXY 2009},
hyperindex = {true},
colorlinks = {true},
linkcolor = {blue},
citecolor = {blue}
]{hyperref} 
\graphicspath{{figs/}}

\input macros.tex
\newcommand{\peq}{+\!\!=}

\begin{document}

\title{Parallel Fast Gauss Transform}  


\author{\IEEEauthorblockN{Rahul S. Sampath}
\IEEEauthorblockA{Oak Ridge National Laboratory\\
       Oak Ridge, TN 37831\\
       Email: sampathrs@ornl.gov} 

\and
\IEEEauthorblockN{Hari Sundar}
\IEEEauthorblockA{Siemens Corporate Research \\
       Princeton, NJ 08540 \\
       Email: hari.sundar@siemens.com}

\and
\IEEEauthorblockN{Shravan Veerapaneni}
\IEEEauthorblockA{New York University \\
       New York, NY 10012 \\
       Email: shravan@cims.nyu.edu}
}
\date{}
\maketitle

\begin{abstract}
\input abstract.tex
\end{abstract}

\section{Introduction}  \label{s:intro}
Gauss transform is one of several discrete spatial transforms of the form 
%
\beq F(x_j) = \sum_{k=1}^N \kernel f_k \quad \text{at} \quad \{ x_j \, | \, j = 1,...,M \} \, , \label{gt} \eeq
\[\text{where} \quad x_j, y_k \in \mathbb{R}^d. \]
%
The kernel $G_\delta$ is a smooth exponentially decaying function in both the physical
and Fourier domains. The parameter $\delta$ controls how rapidly the kernel decays.  
In the Gauss transform case, $\kernel = e^{-\frac{\|x_j - y_k\|^2}{\delta}}$.  We call the
points $x$ as targets and $y$ as sources.  

Discrete sums of the form (\ref{gt}) are encountered in a variety of disciplines including computational physics, machine learning, computational finance and computer graphics. If the kernel decays rapidly, one can simply truncate the sum to a few neighboring sources at each target. However, in many important applications, this is not the case and computing these sums directly takes $\bigO(NM)$ time. 

Starting from the earlier work of Greengard and Strain \cite{fgt}, several sequential algorithms have 
been proposed (e.g.,  \cite{greengard98, duraiswami03, tausch09, fggt}) to reduce the 
cost to an optimal $\bigO(N+M)$. However, to our knowledge, there have been no parallel implementations to date. 

\paragraph{Contributions} Our a
The main contributions of this work are given below.
\begin{itemize} 
%
\item We present a first ever parallel implementation of the fast gauss transform for an 
uniform distribution of points. We incorporate the accelerations introduced in \cite{fggt} for tensor product grids. 
Thereby, the complexity constants in our scheme scale linearly with the number of dimensions 
as opposed to exponential growth. 
%
\item We present a novel scheme for the translation of plane wave expansions; this is one
of the steps in the sequential fast gauss transform algorithm. This new scheme reduces the 
computation and storage costs compared to the previous implementations, especially for highly
non-uniform point distributions.
%
\item We extend our schemes to nonuniform distributions. OCTREES.

\item The cost FGT grows for smaller values of $\delta$. In the nonuniform case, 
We present a scheme that scales well in all ranges of the parameters. It is an extension of the tree-splitting scheme proposed in \cite{veerapaneni08} for computing continuous Gauss transforms. 
%
\end{itemize}

The rest of the paper is organized as follows. In Section \ref{sc:fgt}, we give a high-level description of the sequential FGT algorithm for uniform point distributions. We introduce a novel translation scheme in Section \ref{sc:sweep}. In the uniform distribution case, our scheme is slightly (nearly twice) more expensive compared to sweeping algorithm introduced in \cite{greengard98}. In the nonuniform case, our scheme significantly improves the storage and computational costs. We also discuss its parallel implementation. In Section \ref{sc:nonuniform}, we will present our sequential and parallel algorithms for nonuniform distributions. Finally, in Section \ref{sc:results}, we will present scalability results of our algorithm using a variety of test cases. 

\section{Overview of FGT} 
\label{sc:fgt}
For simplicity, we assume that the points are uniformly distributed and that they reside within a unit cube.
The design of fast algorithms for (\ref{gt}) is strongly dependent on three independent parameter viz., number of sources $N$, the bandwidth $\delta$ and desired accuracy $\epsilon$. A Gaussian centered at a source location interacts with targets that are within its support. If there are fewer targets than a threshold value $n^*$, we use a simple {\em truncation algorithm}, otherwise, we use a {\em expansion algorithm}. The threshold  value depends on all three independent parameters and we will discuss its choice after introducing both algorithms. 

\subsection{Truncation algorithm} 
Since the kernel in (\ref{gt}) decays exponentially, we can simply truncate the sum to
%
\beq F(x_j) = \sum_{y_k \in \mathcal{I}[x_j]} \kernel f_k \eeq
%
where $\mathcal{I}[x_j]$ is the interaction list which includes all the sources that are within a distance $\sqrt{\delta \ln (1/\epsilon)}$. Beyond this distance, a Gaussian centered at $x_j$ decays below $\epsilon$. The complexity of this algorithm is $\bigO(N \sqrt{\delta \ln(1/\epsilon)})$. This is a common technique used in the graphics community. When $\delta$ is large and/or high accuracy is required, the cost of this algorithm grows quadratically.  

\subsection{Expansion algorithm}
There are two variations of FGT: one based on hermite expansion \cite{fgt} and another based on plane-wave expansion \cite{greengard98}. The former has lower expansion costs while the latter has lower translation costs. The latter version is further improved in \cite{fggt} for volumetric data. Our implementation is based on \cite{fggt} and we summarize it here. 

The central is the finite-term plane-wave representation of the kernel,
\beq G_\delta(\norm{x_j - y_k}) \approx \sum_{|k| \leq p} \hat{G}(k) e^{i \lambda k \cdot (x_j - y_k)}, \quad \lambda = \frac{L}{p\sqrt{\delta}}\eeq
where $k = (k_1, k_2, k_3)$ and the parameters $p$ and $L$ are determined by the required precision. $\hat{G}$ is the discrete Fourier transform of the kernel. For the Gaussian kernel, we have 
\beq \hat{G}(k) = \left(\frac{L }{2p\sqrt{\pi}}\right)^3 e^{-\frac{\lambda^2 |k|^2 \delta}{4}}.\eeq

The algorithm begins by partitioning the domain into uniform boxes of size $\sqrt{\delta}$ each. A Gaussian located at the center of a box $B$ decays below $\epsilon$ beyond a fixed number of boxes. We call these boxes the interaction list of $B$, denoted by $\mathcal{I}[B]$. 

In the fast algorithm, a target point $x$ receives information from a source point $y$ via 
\begin{enumerate}
\item{S2W:} The influence of all the sources in a box $B$ is condensed into a plane wave expansion.
            \beq w_k = \eeq
\item{W2L:} The plane wave expansion of each box is transmitted to all the boxes in its interaction list. 
            \beq v_k = \eeq
\item{L2T:} The local plane wave expansion is evaluated at the target locations. 
\end{enumerate} 

When there are significantly more number of sources in each box, it is easy see why we


%

\subsection{A novel scheme for translation} 
Once the wave expansions are formed at all the FGT boxes, the next step is to form each of their local expansions. Since the FGT boxes are of size $\bigO(\sqrt{\delta})$, only a fixed number of surrounding boxes contribute to the local expansion of a box $B$. We shall call these set of boxes as its {\em interaction list}, denoted by $\mathcal{I} [B]$. A {\em direct scheme} forms the local expansion by simply visiting all the boxes in $\mathcal{I}[B]$ and translating their wave expansions. After initializing the local expansions $\{ v_k \quad \forall \quad |k| \leq p \}$ to zero, the pseudo-code for the direct scheme is:
%
{\tt
\begin{algorithmic}
\STATE {\sc Direct scheme}
  \FOR {each $C \in \mathcal{I}[B]$}
           \STATE $ v_k^B \quad += e^{i z_k \cdot(c^B - c^C)/\sqrt{\delta}} w_k^C \quad \forall \quad |k| \leq p$
       \ENDFOR
\STATE
\end{algorithmic}
}
%
Assuming the size of $\mathcal{I}[B]$ is $K^3 - 1$, this algorithm requires $\bigO(K^3 p^3 N_B)$ work to form local expansions at all the boxes. 
%
\section{Translation scheme} \label{sc:sweep}
\input sweep.tex

Based on the cost of the expansion based algorithm, we 

{\tt
\begin{algorithmic}
\STATE
  \IF {$N \sqrt{\delta \ln \left( \frac{1}{\epsilon} \right)} < \frac{1}{2}(2p)^3 $}
     \STATE Use truncation algorithm 
  \ELSE 
     \STATE Use expansion based algorithm
  \ENDIF
\STATE
\end{algorithmic}
}



\section{Nonuniform distributions} 
\label{sc:nonuniform}

{\bf MOTIVATION}

In the uniform distribution case, we can precisely estimate the threshold $n_{th}$ that decides whether one should use the truncation algorithm or the expansion based algorithm. However, when the source and target distributions are highly nonuniform, as is the case in most practical applications, it is not straightforward. For example, when we superimpose a regular grid structure of FGT on a nonuniform distribution, some boxes will have lot of points while some are almost empty. 

%

{\bf OCTREES}
We assume sources and targets are the same for simplicity. We also assume that the octree is constructed so that there are no more than a fixed number of points in each leaf node. 

\begin{algorithm}[!h]
\caption{{\em Tree Splitting}}
{\tt
\begin{algorithmic}
\STATE
  \FOR {each leaf node $\ell$}
      \IF {$|\ell| > \sqrt{\delta}$}
          \STATE assign $\ell$ to $T_d$ (direct or truncation based)
      \ELSE
          \STATE assign $\ell$ to $T_e$ (expansion based)
      \ENDIF
  \ENDFOR
\STATE
\end{algorithmic}
}
\end{algorithm}


\begin{algorithm}[!h]
\caption{\em FGT on a split tree}
%\label{a:ofgt}
{\tt
\begin{algorithmic}
\STATE
%\STATE {\sc FGT on a split Octree}
  \FOR {each $\ell \in T_e$}
      \STATE (S2W) Add contribution of point in $\ell$ to the FGT box it belongs
  \ENDFOR
  \STATE
  \STATE (W2L) Form local expansions using sweeping
  \STATE 

  \STATE {\sc Evaluate the effect of all sources in $T_d$}
  \FOR {each $\ell \in T_d$}
       \FOR {$x \in \ell$}
          \STATE Add contribution of $x$ to all the target boxes (in $T_e$) and target points (in $T_d$)     
       \ENDFOR  
  \ENDFOR
  
  \STATE 
  \STATE {\sc Evaluate the effect of all sources in $T_e$}
     \FOR {each FGT box $B \in T_e$}
        \STATE Use local expansion for target points within $B$ 
        \STATE
        \STATE Use wave expansion for target points in $T_e$ that are within its interaction list
     \ENDFOR  
\STATE
\end{algorithmic}
}
\end{algorithm}

{\em [In the parallel case, the interaction between $T_d$ and $T_e$ can be done while processors are communicating other info.] }



\section{Results}
\label{sc:results}
\begin{table}[h!]
\small
\begin{center}
\begin{tabular}{|c|c|c|c|c|}
\hline
Operation & Max. Time & Avg. Time & Max. Flops & Avg. Flops\\
\hline
S2W & & & & \\
\hline
W2L & & & & \\
\hline
L2T & & & & \\
\hline
Total & & & & \\
\hline
\end{tabular}
\end{center}
\caption{{\rm {\footnotesize Timings on 37,268 ($32^3$) processors on Jaguar. The point distribution is uniform random, the parameter $\delta = 8 \times 10^{-4}$ and precision $\epsilon = 10^{-6}$}}. Each processor has a million points.}}
\label{t:scaling}
\end{table}

\begin{figure*}
	\begin{center}
%		\subfigure[uniform]{\scalebox{1}{\input{figs/uniform_fixed.tex}}}
%		\gap
%		\subfigure[ellipse]{\scalebox{1}{\input{figs/ellipse_fixed.tex}}}
	\end{center}
\caption{Strong scaling on Jaguar. The problem size is fixed to 1 billion points and the parameters are the same as defined in Table \ref{t:scaling}. [Rahul: Try varying the num. of processors from 16 to atleast 8192]}
\label{f:fixed}
\end{figure*}

\begin{figure*}
	\begin{center}
%		\subfigure[uniform]{\scalebox{1}{\input{figs/uniform_iso.tex}}}
%		\gap
%		\subfigure[ellipse]{\scalebox{1}{\input{figs/ellipse_iso.tex}}}
	\end{center}
\caption{Weak scaling on Jaguar. Number of points per proceesor is fixed at 1 million (so $N = 10^6 \times n_p$) and the parameter $\delta = \frac{10}{N^{1/3}}$. [Rahul: Try varying the num. of processors from 16 to atleast 8192]}
\label{f:iso}
\end{figure*}



\section{Conclusions}

%\texbf{Periodic kernels.} If the kernel $G_\delta(x,y)$ is periodic in the unit cube, the fast Fourier transform (FFT) can be used for the fast computation of (\ref{gt}) assuming regular grids. 
Several acceleration techniques for forming and evaluating plane wave expansion were introduced in \cite{fggt}. We will incorporate these in our final submission. When the sources come from a tensor product grid in each octant, the expansion constants can be reduced from exponential to linear in dimension. In the sequential case, this speed-up will make the cost of the algorithm comparable to that of fast Fourier transform (FFT). Since the parallel performance of FFT has been sub-optimal till date, we believe that our parallel FGT would be the method of choice even for regular grids. 


{\em
A few assumptions to simplify our implementation: 
\begin{enumerate}
\item Each processor has finite number of FGT boxes. If we also care for the contrary, we would have a box that spans across processors and hence S2W and L2T also need to parallelized. Not a big deal, but simplifies our job for now. 
\item We will assume that $\delta = 2^{-n}$ for some even number n and FGT box size $h = \sqrt{\delta}$. This will help us in reducing book-keeping: otherwise, in the octree case, we will have FGT boxes cutting across leaf nodes. 
\end{enumerate}
}

\section*{Acknowledgments}
\input acknowledge.tex

\bibliography{sc10}
\bibliographystyle{siam}
\end{document} 
