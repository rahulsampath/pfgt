Gauss transform is one of several discrete spatial transforms of the form 
%
\beq F(x_j) = \sum_{k=1}^N \kernel f(y_k) \quad \text{at} \quad \{ x_j \, | \, j = 1,...,M \} \, , \label{gt} \eeq
\[\text{where} \quad x_j, y_k \in \mathbb{R}^d. \]
%
The kernel $G_\delta$ is a smooth exponentially decaying function in both the physical
and Fourier domains. The parameter $\delta$ controls how rapidly the kernel decays.  
In the Gauss transform case, $\kernel = e^{-\frac{\|x_j - y_k\|^2}{\delta}}$.  We call the
points $x$ as targets and $y$ as sources.  

Discrete sums of the form (\ref{gt}) are encountered in a variety of disciplines including computational physics, machine learning, computational finance and computer graphics. If the kernel decays rapidly, one can simply truncate the sum to a few neighboring sources at each target. However, in many important applications, this is not the case and computing these sums directly takes $\bigO(NM)$ time. 

Starting from the earlier work of Greengard and Strain \cite{fgt}, several sequential algorithms have 
been proposed (e.g.,  \cite{greengard98, duraiswami03, tausch09, fggt}) to reduce the 
cost to an optimal $\bigO(N+M)$. However, to our knowledge, there have been no parallel implementations to date. 

\paragraph{Contributions} The main contributions of this work are given below.
\begin{itemize} 
%
\item We present a first ever parallel implementation of the fast gauss transform for an 
uniform distribution of points. We incorporate the accelerations introduced in \cite{fggt} for tensor product grids. 
Thereby, the complexity constants in our scheme scale linearly with the number of dimensions 
as opposed to exponential growth. 
%
\item We present a novel scheme for the translation of plane wave expansions; this is one
of the steps in the sequential fast gauss transform algorithm. This new scheme reduces the 
computation and storage costs compared to the previous implementations, especially for highly
non-uniform point distributions.
%
\item We extend our schemes to nonuniform distributions. OCTREES.

\item The cost FGT grows for smaller values of $\delta$. In the nonuniform case, 
We present a scheme that scales well in all ranges of the parameters. It is an extension of the tree-splitting scheme proposed in \cite{veerapaneni08} for computing continuous Gauss transforms. 
%
\end{itemize}

The rest of the paper is organized as follows. In Section \ref{sc:fgt}, we give a high-level description of FGT for uniform point distributions and discuss its parallelization. We introduce a novel translation scheme in Section \ref{sc:sweep}. This scheme is slightly (nearly twice) more expensive compared to the {\em sweeping algorithm} introduced in \cite{greengard98} but it significantly improves the storage cost for nonuniform distributions. We also discuss its parallel implementation. In Section \ref{sc:nonuniform}, we will present our octree-based sequential and parallel algorithms for nonuniform distributions. Finally, in Section \ref{sc:results}, we will present scalability results of our algorithm using a variety of test cases. 

