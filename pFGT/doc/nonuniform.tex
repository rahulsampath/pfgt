In the uniform distribution case, one can precisely estimate the threshold $n^*$ that determines whether one should use the truncation or the expansion algorithm. However, when the source and target distributions are highly nonuniform, as is the case in most practical applications, it is not straightforward. For example, when we superimpose a regular grid structure of FGT on a nonuniform distribution, some boxes will have lot of points while some are almost empty. 


In this section, we present an algorithm for nonuniform distributions based on the ideas introduced in \cite{veerapaneni08}. 
We assume sources and targets are the same for simplicity. Similar to the uniform case, we intend to switch between truncation and expansion algorithms to achieve optimal comlexity. Unlike the uniform case, the criterion here will also be based on the distribution of sources apart from $N$, $\delta$ and $\epsilon$. To enable us to identify the threshold, we build an {\em octree} so that there are no more than a fixed number of points in each leaf node. [OCTREE INTRO]
%
%\footnote{The original FGT \cite{fgt} handles nonuniform distributions without using any sort of hierarchical data-structures. In principle, one can extend these ideas to the present context. However there are certain advantages of using octrees both at the implementational and algorithmic levels. [NEED TO EXPAND]}. 
%
%

Now let us consider two extreme cases. If the size of a leaf $\ell$ denoted by $|\ell|$ is much larger than that of FGT box, then converting the sources into plane-wave expansions is futile because there are fewer sources per box. On the other hand, when the leaf nodes are much smaller than the FGT box, then there will be large number of sources within the box covering them and hence it is desirable to convert them to plane-wave expansions. Inspired by this observation, we split the octree as follows. 
%
{\tt
\begin{algorithmic}
\STATE
  \FOR {each leaf node $\ell$}
      \IF {$|\ell| > c\sqrt{\delta}$}
          \STATE assign $\ell$ to $T_d$ (direct tree)
      \ELSE
          \STATE assign $\ell$ to $T_e$ (expand tree)
      \ENDIF
  \ENDFOR
\STATE
\end{algorithmic}
}
%
Here $c$ is constant which is determined by experimentation. Apart from the steps discussed in Section \ref{sc:fgt}, we need extra ones to model the interaction between the trees. We describe them next.  

First, we visit all the boxes $B \in T_e$ and form the plane-wave expansions from the sources (S2W step). Now, we have all the information necessary to compute (\ref{gt}) at all the targets in $T_d$. We proceed in the following two steps.
\begin{description}
\item[\textbf{W2D}] We first evaluate the influence of sources in $T_e$ on targets in $T_d$. Each box $B \in T_e$ visits every target point $x \in T_d$ in its interaction list and evaluates the Gauss transform using
%
\beq F(x) \, +=\, \sum_{|k| \leq p} e^{-\frac{\norm{z_k}^2}{4}} e^{i z_k \cdot (x - c^B)/\sqrt{\delta}} w_k \eeq
%
\item[\textbf{D2D}] Here, we compute the influence of sources in $T_d$ on targets in $T_d$. Each source $y \in T_d$ visits every target $x \in \mathcal[y]$ and updates the potential at $x$ as 
%  
\beq F(x) \,+=\, e^{-\norm{x - y}^2/\sqrt{\delta}} f(y) \eeq
%
\end{description}
%
Until now we have evaluated (\ref{gt}) at all the targets in $T_d$. Now, only the targets in $T_e$ remain. First, we execute the W2L step and convert all the plane-wave expansions to local expansions at target boxes by using the algorithms of Section \ref{sc:sweep}. This accounts for all the sources in $T_e$. 

\begin{description}
\item[\textbf{D2L}] This step is a dual of the W2D. Here, we compute the influence of sources in $T_d$ on targets in $T_e$ by visiting every box $D \in T_e$ in the interaction list of a source $y \in T_d$ and modifying the local expansion as 
%
\beq v_k  \,+=\, \, f(y) e^{i z_k \cdot (c^D - y)/\sqrt{\delta}} \eeq
\end{description}
Finally, we visit each box $D \in T_e$ and compute (\ref{gt}) at all its constituent target points by evaluating the local expansion (L2T step). We summarize all the steps in Algorithm \ref{a:ofgt}.  

\begin{algorithm}[!h]
\caption{\em FGT on a split tree}
\label{a:ofgt}
{\tt
\begin{algorithmic}
\STATE
%\STATE {\sc FGT on a split Octree}
  \FOR {each $\ell \in T_e$}
      \STATE (S2W) Add contribution of point in $\ell$ to the FGT box it belongs
  \ENDFOR
  \STATE W2D
  \STATE D2D 
  \STATE (W2L) Form local expansions using sweeping
  \STATE 

  \STATE {\sc Evaluate the effect of all sources in $T_d$}
  \FOR {each $\ell \in T_d$}
       \FOR {$x \in \ell$}
          \STATE (D2L) Add contribution of $x$ to all the target boxes (in $T_e$) 
       \ENDFOR  
  \ENDFOR
  
  \STATE 
  \STATE {\sc Evaluate the effect of all sources in $T_e$}
     \FOR {each FGT box $B \in T_e$}
        \STATE Use local expansion for target points within $B$ 
        \STATE
        \STATE Use wave expansion for target points in $T_e$ that are within its interaction list
     \ENDFOR  
\STATE
\end{algorithmic}
}
\end{algorithm}

